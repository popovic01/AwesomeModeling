\section{Deployment}
\label{ch:depl}

In this project, we utilized Docker Compose to manage the deployment of a
microservices architecture. Docker Compose allows us to define and run
multi-container Docker applications, ensuring a consistent and reproducible
deployment environment.

Below we present some considerations for
deploying our services using Docker Compose:

\subsection{Services}
Since our application is based on a microservices architecture, we decided
to put every service inside a Docker container.
We declared the following services in the docker compose configuration:
\begin{itemize}
        \item \textbf{MongoDB} that contains a process running the MongoDB server.
        \item \textbf{ElasticSearch} containing a running ElasticSearch server.
        \item \textbf{RabbitMQ} for hosting the message queue.
        \item \textbf{SpringBoot} that handles the client REST requests and
                communicates with the other microservices.
        \item \textbf{Worker} that waits for messages to be published in the
                queue and executes jobs.
\end{itemize}
Some additional services:
\begin{itemize}
        \item \textbf{Kibana}, a user interface for Elastic, used as a debugging
                tool.
        \item \textbf{Mongo Express}, a user interface for MongoDB, useful
                for navigating the collections with an user interface.
        \item \textbf{FrontEnd}, a nginx process used to serve to the user
                a small and lightweight Angular applications. This is the
                way a "final" user could use the application.
\end{itemize}

It is important to note that we used the following configuration:
\begin{lstlisting}[frame=single,caption=Replicas of a services,label=replicas]
deploy:
  mode: replicated
  replicas: 2
\end{lstlisting}

In this way docker compose will create more than one container for the same
image. We set this value to 2 because we are in a "development" environment.
If this application was being run on a more powerful machine (or even better
on a cluster of machines) we could use much higher values.
This parameter it is useful since it will allow our application to process more
than one job at the same time, using the queue as a load balancer.

Health checks are crucial for the correct execution of services. They make sure
that each service is operational before other dependent services start. MongoDB and
RabbitMQ have specific health check commands configured.
In this way, for example, we will avoid that a worker (that depends on the message
queue) is started before the queue is operational.

The \textit{depends\_on} attribute in Docker Compose ensures that services start
in the correct order. For example, the Python application waits for MongoDB and
RabbitMQ to be healthy before it starts, ensuring that all required
dependencies are available.

The full configuration for the docker compose can be seen in the
\textit{docker-compose.yml} file.

\subsection{Volumes}
In order to avoid losing all the data when the container is destroyed, we created
2 volumes. One will be mounted in the data folder of the MongoDB container and
the other one in the data folder of the ElasticSearch container. In this way,
both the database and Elastic will store their data in a volume.
