\chapter{Methodology}\label{ch:met}

\section{Technologies used}
Firstly, the used technologies such as Spring, Mongo and Mallet will be briefly described in order
to easier follow project description later.
\subsection{Spring}
The Spring Framework is a comprehensive and widely used Java-based framework for building
enterprise-level applications. It provides a robust infrastructure for developing Java applications.
It simplifies the development of complex applications by promoting good design practices and
offering a suite of tools and libraries for building web applications, microservices, and
data-driven solutions. Additionally, Spring's modular architecture and extensive ecosystem
allow developers to use only the components they need, making it highly flexible and scalable.
\cite{spring}

\subsection{MongoDB}
MongoDB is a popular open-source, document-oriented NoSQL database designed for scalability,
flexibility, and performance. It stores data in flexible, JSON-like documents, allowing for varied
and dynamic data structures without requiring a fixed schema. MongoDB is known for its ability to handle large
volumes of data and its powerful querying and indexing capabilities. It supports a wide range of applications.
\cite{mongodb}

\subsection{Python}
Python is a versatile and high-level programming language known for its readability, simplicity, and extensive
standard library. It supports multiple programming paradigms, including procedural, object-oriented, and
functional programming. Python's dynamic typing and interpreted nature make it an excellent choice for rapid
application development. Python is also widely used in developing microservices due to its ease of use and
speed of development. \cite{python}

\subsection{Elasticsearch}
Elasticsearch is a powerful, open-source search and analytics engine designed for horizontal scalability,
reliability, and real-time search capabilities. It is built on Apache Lucene and provides a distributed, text
search engine with an HTTP web interface and schema-free JSON documents. It is commonly used for log and event
data analysis, full-text search, and real-time analytics due to its high performance, flexibility, and ability
to handle large volumes of data across distributed systems. \cite{elastic}

\subsection{Mallet}
MALLET (MAchine Learning for LanguagE Toolkit) is a Java-based open-source toolkit for statistical natural
language processing, particularly renowned for its implementations of topic modeling algorithms. It supports
various algorithms for discovering latent topics in large collections of text documents. MALLET provides tools
for preprocessing textual data, training topic models, and evaluating model performance. \cite{mallet}

\subsection{RabbitMQ}
RabbitMQ is a powerful open-source message broker that implements the Advanced Message Queuing Protocol (AMQP).
It enables seamless communication between distributed systems by acting as a mediator that facilitates the
reliable transfer of messages between applications and services. It supports various messaging patterns such
as point-to-point, publish/subscribe, and request/response. It is widely used in microservices architectures,
IoT applications, and asynchronous communication scenarios where decoupling and reliability are crucial.
\cite{rabbitmq}