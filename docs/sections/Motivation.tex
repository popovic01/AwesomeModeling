\section{Motivation for the specific design choices}
\label{ch:motivation}

In this chapter we will explain design choices we made and the reasons for that.

During the development of the Spring service, we followed the \textit{Separation of Concerns} principle, which is a
fundamental principle in software engineering and design. It is used to separate an application into units with
minimal overlapping between the functions of the individual units \cite{geeksforgeeks:soc}. This is done by splitting
the logic into three different layers: \textit{controllers}, \textit{services}, and \textit{repositories}.

The \textit{controllers} folder contains a class \textit{QOneController.java} responsible for handling HTTP requests
and mapping them to the appropriate service methods. This keeps the request handling logic separate from the business
logic and the data access logic.

The \textit{services} folder holds the business logic of the service in our case, \textit{MalletService.java}. The
\textit{repositories} folder contains the data access logic, specifically an interface that extends
\textit{MongoRepository} \textit{ControlRepository.java}. This separation allows us to change the data access layer
without affecting the business logic.

Another good design practice was creating \textit{entities} and \textit{dtos} folders. The \textit{entities} classes
represent the data models or the domain objects. By keeping them in a separate folder, we ensure that the domain logic
is isolated from the rest of the application. The \textit{dtos} folder keeps \textit{Data Transfer Objects} classes
separately, which are used to transfer data between the layers of the application, especially between the client and the server.
