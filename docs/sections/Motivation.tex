\section{Motivation for the specific design choices}
\label{ch:motivation}

\subsection{Overall architecture}
First of all, we decided to detach the indexing and downloading of articles
from the endpoint. This was done for one main reason: the retrieval and indexing
of documents can be a rather long process, and we didn't want to do it during
the API call. For this reason we decided to do this asynchronously. The user
will send the request to the endpoint which is going to insert the job into a
queue. A response is sent to the client immediately, telling it that the request
is going to be processed. In a second moment one of the worker nodes will read
the job request from the queue and will start to process it. Once it is completed
it will write into a control collection in the database that it completed the job
so that, an user sending a request to check the status of the process, can
see that the request is completed.

This can also help us to solve a second problem, \textit{Scalability}.
By having detached the endpoint and the ``worker nodes'' we can have a large number
of them, working on the same queue. The queue will act as a \textit{Load Balancer},
indirectly distributing the jobs to the different workers.

\subsection{API Endpoint (Spring)}
In this chapter we will explain design choices we made and the reasons for that.

During the development of the Spring service, we followed the \textit{Separation of Concerns} principle, which is a
fundamental principle in software engineering and design. It is used to separate an application into units with
minimal overlapping between the functions of the individual units \cite{geeksforgeeks:soc}. This is done by splitting
the logic into three different layers: \textit{controllers}, \textit{services}, and \textit{repositories}.

The \textit{controllers} folder contains a class \textit{QOneController.java} responsible for handling HTTP requests
and mapping them to the appropriate service methods. This keeps the request handling logic separate from the business
logic and the data access logic.

The \textit{services} folder holds the business logic of the service in our case, \textit{MalletService.java}. The
\textit{repositories} folder contains the data access logic, specifically an interface that extends
\textit{MongoRepository} \textit{ControlRepository.java}. This separation allows us to change the data access layer
without affecting the business logic.

Another good design practice was creating \textit{entities} and \textit{dtos} folders. The \textit{entities} classes
represent the data models or the domain objects. By keeping them in a separate folder, we ensure that the domain logic
is isolated from the rest of the application. The \textit{dtos} folder keeps \textit{Data Transfer Objects} classes
separately, which are used to transfer data between the layers of the application, especially between the client and the server.

\subsection{Worker node}
Since the worker node needs to interact with the Mongo database and send REST requests
to Elastic to index documents, we decided to develop it using Python.
This allows us to manage json objects with dictionaries and to send HTTP requests
with the \textit{requests} library.
