\section{Testing}
In order to test if the application works as expected, we used 2 different
approaches.

\subsection{Unit Testing}
The first one, \textit{Unit Testing}, was used for the Spring Microservice.
As we discussed in the previous sections, the Spring application contains
a Service that we use to interact with the Mallet library. To test this service
we wrote a specific class, called \texttt{MalletServiceTest}, that checks whether
the calls to the Mallet library work as we expect.

\subsection{End-to-end testing}
End-to-end (E2E) testing is a methodology used to validate the complete and
integrated functionality of an application, ensuring that the entire workflow
behaves as expected from start to finish.
Differently from unit testing, which focuses on individual components or
services, E2E testing simulates real user scenarios and interactions with the
system, covering all interconnected parts.

For our application, we designed a E2E test that covers a typical workflow
the final user could go through.
In order to accomplish this, we wrote a python script that, with the use of
the \textit{requests} library, interacts with the REST endpoint and checks
if everything works.

The script goes through the following steps:
\begin{itemize}
        \item \textbf{Setup of the application}: for this step we use the
                command that deploys the application using docker compose.
        \item \textbf{Creation of Q1}: we create a Q1.
        \item \textbf{Await for the completion}: the script waits (by sending
                a request every few seconds) for the workers to download and
                index the collection.
        \item \textbf{Generating topics (Q2)}: the script sends a request for
                the generation of topics based on a second query (Q2). The script
                then checks whether the api returned $k$ valid topics.
        \item \textbf{Shutting down the application}: we use the docker
                compose command that stops and deletes all the container associated
                with the application.
\end{itemize}
